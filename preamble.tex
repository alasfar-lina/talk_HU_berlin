\usepackage{pdfpages}
\usepackage{xspace}
\usepackage{morefloats,subfig,afterpage,slashed,mathrsfs,amssymb}
\usepackage{amsfonts}
\usepackage{graphicx}
\usepackage{listings}
\usepackage{epstopdf}
\usepackage{multirow,multicol}
\usepackage{float}  
\usepackage{tikz}
\usepackage{cite} 
\usepackage{amsfonts}
\usepackage{amsmath, amssymb, } %amsthm, latexsym, amscd, enumerate, MnSymbol,bbm, etex,nicefrac,mathrsfs}
% \newcommand\hmmax{0} % default 3
% % \newcommand\bmmax{0} % default 4
% \usepackage{bm}
%% Set font 
%\setromanfont{Cambria}
%\setsansfont{Calibri}
%\setmonofont{Consolas}
%\setmathfont{Cambria Math}

\usepackage[  %Paket Glossaries laden, muss nach Hyperref geladen werden!
xindy,
nonumberlist, % keine Seitenzahlen anzeigen           
acronym,      % ein Abkürzungsverzeichnis erstellen
toc           % Einträge im Inhaltsverzeichnis
] 
{glossaries}
\everymath{\displaystyle}
%\usepackage{geometry}
\usepackage{pgfplotstable}
\usepackage{array}
\usepackage{booktabs}
\usepackage{pgfplots, pgfplotstable}
\usepgfplotslibrary{dateplot}
%\geometry{
%	paper = a4paper , % Change to letterpaper for US letter
%	inner = \dimexpr0.5in+33pt\relax , % Inner margin
%	outer = \dimexpr0.5in+22pt\relax , % Outer margin
%	top = \dimexpr1in+12pt+24pt\relax , % Top margin
%	bottom = \dimexpr 1in+24pt\relax, % Bottom margin
%	marginpar = 51pt ,
%	marginparsep = 17pt ,
%	foot = 24pt ,
%	headsep = 24pt
%}
%\usepackage{showkeys}
%\usepackage[]{latexsym}
%\usepackage{}
%\graphicspath{./figures/}
%% Using Babel allows other languages to be used and mixed-in easily
%\usepackage[ngerman,english]{babel}


%% Citation system tweaks
% \let\@OldCite\cite
% \renewcommand{\cite}[1]{\mbox{\!\!\!\@OldCite{#1}}}

%% Maths
% TODO: rework or eliminate maybemath
\usepackage{abmath}
\DeclareRobustCommand{\mymath}[1]{\ensuremath{\maybebmsf{#1}}}
\DeclareRobustCommand{\parenths}[1]{\mymath{\left({#1}\right)}\xspace}
\DeclareRobustCommand{\braces}[1]{\mymath{\left\{{#1}\right\}}\xspace}
% \DeclareRobustCommand{\angles}[1]{\mymath{\left\langle{#1}\right\rangle}\xspace}
\DeclareRobustCommand{\sqbracs}[1]{\mymath{\left[{#1}\right]}\xspace}
\DeclareRobustCommand{\mods}[1]{\mymath{\left\lvert{#1}\right\rvert}\xspace}
\DeclareRobustCommand{\modsq}[1]{\mymath{\mods{#1}^2}\xspace}
 \DeclareRobustCommand{\dblmods}[1]{\mymath{\left\lVert{#1}\right\rVert}\xspace}
 \DeclareRobustCommand{\expOf}[1]{\mymath{\exp{\!\parenths{#1}}}\xspace}
 \DeclareRobustCommand{\eexp}[1]{e^{#1}\xspace}
 \DeclareRobustCommand{\plusquad}{\mymath{\oplus}\xspace}
 \DeclareRobustCommand{\logOf}[1]{\mymath{\log\!\parenths{#1}}\xspace}
 \DeclareRobustCommand{\lnOf}[1]{\mymath{\ln\!\parenths{#1}}\xspace}
\DeclareRobustCommand{\ofOrder}[1]{\mymath{\mathcal{O}\parenths{#1}}\xspace}
\DeclareRobustCommand{\SOgroup}[1]{\mymath{\mathup{SO}\parenths{#1}}\xspace}
\DeclareRobustCommand{\SUgroup}[1]{\mymath{\mathup{SU}\parenths{#1}}\xspace}
\DeclareRobustCommand{\Ugroup}[1]{\mymath{\mathup{U}\parenths{#1}}\xspace}
%\DeclareRobustCommand{\I}[1]{\mymath{\mathrm{i}}\xspace}
\DeclareRobustCommand{\colvector}[1]{\mymath{\begin{pmatrix}#1\end{pmatrix}}\xspace}
\DeclareRobustCommand{\Rate}{\mymath{\Gamma}\xspace}
\DeclareRobustCommand{\RateOf}[1]{\mymath{\Gamma}\parenths{#1}\xspace}
\DeclareRobustCommand{\pt}{p_{T}}
\DeclareRobustCommand{\fOf}[1]{\mymath{f_{\text{#1}}\xspace}}
\DeclareRobustCommand{\ToThe}[2]{\mymath{#1\cdot 10^{#2}}}
\DeclareRobustCommand{\sqrts}{\sqrt{s}\xspace}
\DeclareRobustCommand{\intlum}{\int \mathscr L dt}
\newcommand{\E}[1]{\mbox{$\cdot 10^{#1}$}}
\DeclareRobustCommand{\bra}[1]{\mbox{$\langle\, #1 \mid$}}
\newcommand{\bbra}[1]{\mbox{$\left\langle\, #1 \right\mid$}}
\DeclareRobustCommand{\ket}[1]{\mbox{$\mid #1\,\rangle$}}
\newcommand{\bket}[1]{\mbox{$\left\mid #1\,\right\rangle$}}
\newcommand{\pro}[2]{\mbox{$\langle\, #1 \mid #2\,\rangle$}}
\newcommand{\expec}[1]{\mbox{$\langle\, #1\,\rangle$}}
\newcommand{\expecl}[1]{\mbox{$\left\langle\,\strut\displaystyle{#1}\,\right\rangle$}}
\newcommand{\be}{\begin{eqnarray}}
\newcommand{\ee}{\end{eqnarray}}
\newcommand{\bea}{\begin{eqnarray}}
\newcommand{\eea}{\end{eqnarray}}
\DeclareRobustCommand{\RK}{\mymath{R_{K^{(*)}}} }
\DeclareRobustCommand{\RD}{\mymath{R_{D^{(*)}}} }

%%% High-energy physics stuff
%\usepackage{abhep}
%\usepackage{hepnames}
\usepackage{hepunits}
\DeclareRobustCommand{\arXivCode}[1]{arXiv:#1}
\DeclareRobustCommand{\CP}{\ensuremath{\mathcal{CP}}\xspace}
\DeclareRobustCommand{\CPviolation}{\CP-violation\xspace}
\DeclareRobustCommand{\CPv}{\CPviolation}
\DeclareRobustCommand{\LHCb}{LHCb\xspace}
\DeclareRobustCommand{\LHC}{LHC\xspace}
\DeclareRobustCommand{\LEP}{LEP\xspace}
\DeclareRobustCommand{\CERN}{CERN\xspace}
\DeclareRobustCommand{\VELO}{VELO\xspace}
%\DeclareRobustCommand{\bphysics}{\Pbottom-physics\xspace}
%\DeclareRobustCommand{\bhadron}{\Pbottom-hadron\xspace}
%\DeclareRobustCommand{\Bmeson}{\PB-meson\xspace}
%\DeclareRobustCommand{\bbaryon}{\Pbottom-baryon\xspace}
%\DeclareRobustCommand{\Bdecay}{\PB-decay\xspace}
%\DeclareRobustCommand{\bdecay}{\Pbottom-decay\xspace}
%\DeclareRobustCommand{\BToKPi}{\HepProcess{ \PB \to \PK \Ppi }\xspace}
%\DeclareRobustCommand{\BToPiPi}{\HepProcess{ \PB \to \Ppi \Ppi }\xspace}
%\DeclareRobustCommand{\BToKK}{\HepProcess{ \PB \to \PK \PK }\xspace}
%\DeclareRobustCommand{\BToRhoPi}{\HepProcess{ \PB \to \Prho \Ppi }\xspace}
%\DeclareRobustCommand{\BToRhoRho}{\HepProcess{ \PB \to \Prho \Prho }\xspace}
%\DeclareRobustCommand{\BToKll}{\HepProcess{ \PB \to \PKstar(892) \Plepton \Plepton }\xspace}
%\DeclareRobustCommand{\BToKtt}{\HepProcess{ \PB \to \PKstar(892) \tau^- \tau^+ }\xspace}
%\DeclareRobustCommand{\BToKmm}{\HepProcess{ \PB \to \PKstar(892) \Pmuon  \APmuon  }\xspace}
%\DeclareRobustCommand{\BToKee}{\HepProcess{ \PB \to \PKstar(892) \Pelectron \APelectron }\xspace}
%\DeclareRobustCommand{\X}{\thesismath{X}\xspace}
%\DeclareRobustCommand{\Xbar}{\thesismath{\overline{X}}\xspace}
%\DeclareRobustCommand{\Xzero}{\HepGenParticle{X}{}{0}\xspace}
%\DeclareRobustCommand{\Xzerobar}{\HepGenAntiParticle{X}{}{0}\xspace}
%\DeclareRobustCommand{\epluseminus}{\Ppositron\!\Pelectron\xspace}
%\DeclareRobustCommand{\protonproton}{\Pproton\APantiproton\xspace}

\definecolor{mygreen}{rgb}{0,0.6,0}
\definecolor{mygray}{rgb}{0.5,0.5,0.5}
\definecolor{mymauve}{rgb}{0.58,0,0.82}

\lstset{ %
	backgroundcolor=\color{white},   % choose the background color
	basicstyle=\footnotesize,        % size of fonts used for the code
	breaklines=true,                 % automatic line breaking only at whitespace
	captionpos=b,                    % sets the caption-position to bottom
	commentstyle=\color{mygreen},    % comment style
	escapeinside={\%*}{*)},          % if you want to add LaTeX within your code
	keywordstyle=\color{blue},       % keyword style
	stringstyle=\color{mymauve},     % string literal style
}

%\definecolor{dkgreen}{rgb}{0,0.6,0}
%\definecolor{gray}{rgb}{0.5,0.5,0.5}
%\definecolor{mauve}{rgb}{0.58,0,0.82}
%\lstset{frame=tb,
%	language=c++,
%	aboveskip=3mm,
%	belowskip=3mm,
%	showstringspaces=false,
%	columns=flexible,
%	basicstyle={\small\ttfamily},
%	numbers=none,
%	numberstyle=\tiny\color{gray},
%	keywordstyle=\color{blue},
%	commentstyle=\color{dkgreen},
%	stringstyle=\color{mauve},
%	breaklines=true,
%	breakatwhitespace=true,
%	tabsize=3
%}